   
%\subsubsection{Keyword and Metadata Search}\label{sec:keyword_search}

Our system implements two types of possibilities for querying the database that can be combined. The first option is to query the collection by specifications on metadata. Currently this includes filtering on dates and on types of documents. These specifications are translated to MySQL queries according to which only the documents that fulfill the chosen constrains are retrieved from the database. 

The second option is to specify keywords that can be NE's or other lexical items. In order to perform queries based on keywords we transform each query to a vector representation. This is essentially the same transformation that is done for documents. Query terms and document are are normalized to lowercase and checked for exact matching. In order to determine the relevance of each document to the query the cosine similarity (described in the next section) between the query vector and each document vector is computed. 

Note that if aliasing is enabled, an indirect query expansion is performed for query terms that are NE's. If the representation of the documents is expanded with aliases, a document that does not contain the exact term expressed in the query might match the query because it matches one of the aliases. For example, assume that a document originally contains the NE \lingform{Castro} and that the query is for the NE \lingform{Fidel Castro}. Without aliasing these two terms will not match. However, if aliasing is enabled the document might be boosted with the string \lingform{Fidel Castro} which will lead to matching between previously unrelated terms. 

Combining the two types or queries is translated into a two step procedure: Firstly, documents that do not satisfy the metadata conditions of type and date of the document are filtered out. Secondly, documents are sorted in descending order according to their cosine similarity to the query. If the search is solely on metadata, documents are not sorted. Only the subset of the best results of the size specified by the user proceed to the next stage of inter-document similarity calculations.
