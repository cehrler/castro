While the Stanford CRF-based NER system is highly accurate and recognizes the majority of named entities in any given document, it does not always correctly recognize the full and exact substring corresponding to the named entity in question. For example, the location \lingform{Santiago de Cuba} was often recognized as e.g.\ \lingform{Santiago} \lingform{Santiago de} or \lingform{Santiago de Cuba-is}, and may even be recognized as two NEs--- as \lingform{Santiago} and \lingform{Cuba}:

\begin{table}
\centering
\caption{NE form errors}
\begin{tabular}{l r | l r }\toprule
 \textbf{Form} & \textbf{Relative frequency} & \textbf{Form} & \textbf{Relative frequency} \\
\midrule
\emph{Santiago de Cuba} & & & \\
\midrule
Santiagode Cuba & & & \\
Santiago deCuba & & & \\
SantiagodeCuba & & &\\
Santiago de Cuba-is & & &\\
\bottomrule
\end{tabular}
\end{table}

\begin{table}
\centering
\caption{NE form errors}
\begin{tabular}{l r | l r }\toprule
 \textbf{Form} & \textbf{Relative frequency} & \textbf{Relative frequency} \\
\midrule
\multicolumn{2}{l}{\emph{Cuban}} & \multicolumn{2}{r}{}\\
\multicolumn{2}{l}{\emph{Soviet}} & \multicolumn{2}{r}{}\\
\multicolumn{2}{l}{\emph{French}} & \multicolumn{2}{r}{}\\
\midrule
Cuban-Soviet & & Cuban-French & \\
Soviet-Cuban & & Soviet-French & \\
French-Cuban & & French-Soviet & \\
\bottomrule
\end{tabular}
\end{table}

Similarly, the type of NE recognized may be incorrect. For example, \lingform{Lenin} was recognized as \meta{Location} on more than one occasion. These problems can also occur concurrently: Just as with \lingform{Santiago de Cuba}, \lingform{Fidel Castro} may not be recognized in its entirety as a single NE, and more than once was recognized as two separate NEs of different types: \lingform{Fidel} as \meta{Organization} and \lingform{Castro} as \meta{Person}.

Although the total error rate is small, it is not inconsequential: For example, in one document\footnote{ID 396: \scare{Speech to Textile Workers in Tome}}, there were 37 NEs. Four of the NEs were not recognized at all, one was only recognized partially, and one was recognized as the wrong type. It seems that while the NER is mostly accurate, the errors made in NER are propagated to all downstream components, affecting the end results.
