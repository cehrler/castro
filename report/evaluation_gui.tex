The results of the document similarity visualization are mixed: While there are many relationships represented between documents, the qualitative value of these relationships is irregular.

For example, a query including the NE \code{"Central Committe"} and the keyword
\note{add GENERAL QUERY}

However, querying more specific terms revealed different results: A query was entered using the terms \code{Giron}, and \code{Kennedy}, retrieving the 10 most relevant documents and enabling Chinese Whisper Clustering (set to a maximum of four clusters). Two distinct clusters of documents were produced: One was of documents directly related to the 1961 Bay of Pigs Invasion (a.k.a.\ Playa Gir\'{o}n), e.g.\ speeches on the anniversary of the event, or victory speeches. However, the other cluster was composed largely of documents in which \lingform{Playa Giron} was mentioned but the document was not directly related to the event. Most featured political and/or economic topics as well as the topic of US--Cuban relations or aggression.

While all of these documents could be retrieved with a simple keyword-based search, the clustering of documents into similar themes is novel compared to a simple ranking of documents as seen in traditional keyword-based searching.

However, such results were obtained only with relatively specific NEs--- Contrast the query involving \code{``Giron''} and \code{``Kennedy''} to that involving \code{``Central Committee''} ($\sim$ \code{``Castro''})  and \code{economy}. Additionally, even after entering a relatively specific query, much adjustment of the visualization settings was required to display an graph which was both of practical use to the user and was not overwhelming in size and complexity.