Before discussing concrete examples, it is important to stress that designing a valid evaluation scheme for the whole application is an extremely difficult task. The first challenge that we face concerns the multitude of implemented features and adjustable parameters and their combinations. Accounting for the effect of each adjustment for each functionality is very difficult. Furthermore, results of such evaluations in our setup strongly depend on the set of documents that happened to be retrieved.   

More importantly, the notions of similarity or relatedness are very difficult to translate into clear evaluation schemes when the objects of analysis are documents, and in particular rhetorical speeches that encompass a multitude of topics. A set of documents that seem highly interconnected to one person might look as a random collection of documents to another. Evaluation becomes even more difficult when the set of retrieved documents is large. 

Having said that, we would like to present two examples of possible queries along with their results.
A search for \code{"Central Committee"} was conducted between the years \code{1959} and \code{1995}, retrieving the ten most relevant documents. The resulting graph based on NE similarity showed a large number of 
edges (NE links) between documents. After a qualitative examining of the document contents, it was seemed that the documents were not strongly related. Additionally, activating CWC clustering did not have a significant effect: The CWC clustering algorithm did not discern any distinct density regions in the graph.

However, querying more specific terms revealed different results: A query was entered using the terms \code{Giron} and \code{Kennedy}, retrieving the 10 most relevant documents and enabling Chinese Whisper Clustering (set to a maximum of four clusters). Two distinct clusters of documents were produced: One was of documents directly related to the 1961 Bay of Pigs Invasion (a.k.a.\ Playa Gir\'{o}n), e.g.\ speeches on the anniversary of the event, or victory speeches. The other cluster was composed largely of documents in which \lingform{Playa Giron} was mentioned but the document was not directly related to the event. Most featured political and/or economic topics as well as the topic of US--Cuban relations or aggression.

While all of these documents could be retrieved with a simple keyword-based search, the clustering of documents into similar themes is novel compared to a simple ranking of documents as seen in traditional keyword-based searching.

However, such results were obtained only with relatively specific NEs--- Contrast the query involving \code{Giron} and \code{Kennedy} to that involving \code{"Central Committee"} ($\sim$ \code{Castro})  and \code{economy}. Additionally, even after entering a relatively specific query, much adjustment of the visualization settings was required to display an graph which was both of practical use to the user and was not overwhelming in size and complexity.
