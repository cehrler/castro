\note{Detailed description of the subtasks that your system performs: 
how does it tackle these subtasks? 
why did you choose this approach? 
did you encounter specific problems that you had to solve? 
how did you solve them?}

\subsection {Data}
\label{sec:data}

\subsubsection{Description of the Castro Archive}
\label{sec:description_of_the_castro_archive}
\note{technical details on the database, statistics, typical structure of a document}

\subsubsection{Reading and Storing the Data}
\label{sec:reading_and_storing_the_data}
\note{reading the data, creating MySQL database, creating data model}

At a first step we used a \texttt{shell}\footnote{http://tiswww.case.edu/php/chet/bash/bashtop.html}
script and \texttt{wget}\footnote{http://www.gnu.org/software/wget/} to download the complete castro
speech database. The actual speech data consists of a set of static webpages sorted by year into
folders. In a second processing step we parsed these html-files and extracted the actual content of
each file and its meta-information into textual format. To this end, several \texttt{shell} and
\texttt{python}\footnote{http://www.python.org/} scripts have been written.

The extracted information serves as the basis for the consecutive analysis and processing steps. In
order to provide structured access to the meta-information we developed a
\texttt{perl}\footnote{http://www.perl.org/} script to convert the meta-information for each
document into an entry in a \texttt{mysql}\footnote{http://www.mysql.de/products/enterprise/}
database. For every document, the database provides the following information
\begin{itemize}
 \item{\textsc{Author:} Name of the author of the document.}
 \item{\textsc{Location:} The place the document orginated.}
 \item{\textsc{Headline:} The headline of the document.}
 \item{\textsc{Date:} The date the document orginated.}
 \item{\textsc{ReportDate:} The date th3e content of the document appeared in written from.}
 \item{\textsc{Type:} There are several different types of documents in the databse. E.g.
 \texttt{SPEECH}, \texttt{INTERVIEW}, \texttt{ARTICLE}, etc.}
 \item{\textsc{Header:} Additional meta-information}
 \item{\textsc{Source:} The source where this document appeared in written form the first time.}
\end{itemize}
In addition to this information, we later enriched the database with the extracted named entities
(see \ref{sec:named_entity_recognition}) for each document. The columns \textsc{Persons},
\textsc{Places}, \textsc{Organizations} have been added and each field contains the bulk information
of the identified named entities as a comma separated list. Although this representation is is not
the most elegant and optimal solution, the derived database provides sufficient performance and
usability for the task at hand.

\subsection {Named Entity Recognition}
\label{sec:named_entity_recognition}
Please recall, that our ultimate goal is to come up with a cross-document recommendation system for
the castro speech database. Therefore, we need an appropriate metric on between document
similarity that serves as a basis for the recommendation software.

Our first idea was to use proper \textit{co-reference resolution}, e.g. provided by
\texttt{BART}\footnote{http://www.bart-coref.org/} software \cite{bart}. In this case, document
similarity is a function of the number of co-references that appear between two docuemnts. However,
in the end we decided to drop this approach because of the overall complexity and the tight time
constraints. In general, between document co-reference resolution far from an easy task. The castro
database though has some nice properties that allow for a more straightforward solution. For this
dataset, the domain is rather limited and the number of distinct topics should be small. Hence, we
came up with the idea to base the similarity measure on smoothed co-occurences of named entities.

The next step in the pre-processing chain is thus the execution of named entity recognition followed
by a smooting of the features in the vector space induced by the named entities using kernel
methods (see \ref{sec:string_kernel}) \cite{string_kernel_coref}.

\subsubsection{The Stanford Named Entity Recognizer}
\label{sec:stanford_named_entity_recognizer}
\note{details on the tool}

\subsubsection{Recognizing Named Entities in the Castro Archive}
\label{sec:recognizing_named_enitiies_in_the_castro_archive}
\note{identifying names, locations and organizations in the Castro documents,
input/output examples, evaluation??}

\subsection {Representing Documents in Vector Space}
\label{sec:representing_documents_in_vector_space}

\subsubsection{Indexing and Term Weighting}
\label{sec:indexing_term_weighting}
\note{description of the 4 different representations, tf/tfidf with motivation and
interpretation for each, examples}

\subsubsection {Aliasing with String Kernels}
\label{sec:aliasing_string_kernel}
\note{motivation, description of the algorithm to create similarity matrixes between NE's,
smoothing document representations with Kernel matrix as aliasing}

\subsection {Information Retrieval}
\label{sec:information_retrieval}

\subsubsection{Keyword Search}
\label{sec:keyword_search}
\note{description of the functionality, use of string kernels for query expansion }

\subsubsection{Metadata Search}
\label{sec:metadata_search}
\note{general description of the functionality}

\subsection {Computing Similarity between Documents}
\label{sec:computing_similarity_between_documents}
\note{calculating similarity matrice.}

\subsection {Graphical Representation}
\label{sec:graphical_representation}
\note{MICHAL - EXPAND SECTION WITH STUFF ABOUT THE COMPUTATIONAL, ALGORITHMIC, AND IMPLEMENTATION STUFF.}

So far we have described how the nodes and edges in the graph are obtained. In order to display the graph, a graph layout (two dimensional representation of the graph) must be determined. If our vector representation of the document was only two dimensional then it would be very straightforward to obtain the graph layout
by just using the two feature values as x and y coordinates. However the document in the described model is represented by the vector of several thousands of dimensions. This problem is usually not solved by direct mapping from high dimensional vector space to the low dimensional vector space. There are algorithm that solve the problem of graph layouting directly by using only the properties of the underlying graph (nodes and edges together with their weights).

The layouting algorithm that was used in our project belongs to the popular class of the layouting algorithms called force-based layout algorithms. In this class of algorithms physical modelling is used. Spring is considered to be string and the node is considered to be electrically charged particle (all the nodes have the same charge). The optimal length of the edge/spring is set in advance. if the length of the edge exceeds the optimal value, then there is a power between connected nodes trying to push them together. If the distance of two nodes is lower than the predefined repulsion distance, then there is a repulsion power trying to push them apart. In each iteration of the layouting algorithm nodes are moved acording to the resultant force that takes effect on them. After some time the position of the nodes becomes stationary, the layout achieves so called equilibrum state.

The class of forse based layout algorithms groups together various different algorithms with different properties. The idea of a metaphor between nodes and edges of the graph and the real world object that interact together and the step by step simulation are their common properties. In our project we used the very basic algorithm as it is described in the previous article.

For the purpose of graph layouting, graph painting and user-graph interaction, we used Jung - java universal network/graph library. It provides great flexibility of adjusting the graph appearence. Several transformers that affect the node label, label and edge appearence can be plugged together to result in the aesthetically pleasent appearence. There are also predefined classes that allows the easy interaction with the user. Functionalities of graph translating, zooming, node picking, node transposition and many others can be achieved with minimal programming effort. The strong point of the library is it's design based on general template classes with the possibility to include several pluggable component such as various renderers and transformers that affect the resulting appearance of the graph. 

However there were few aspects of the jung library, that we found annoying. We found it difficult to achieve some low level functionalities such as retrieving the changing position of fixed staring point in the actual graphical representation, when the graph translating is performed. Also several operation are done in quite inefficient way. For example in order to change the color of one node, the whole graph must be redrawn asking all the vizualization transformer on the appearence of each vertex or node. When we used a default implementation of a jung visualization class, the application consumed more then 50\% of the processor capacity even though there was no user interaction or computation task. Then we found out that the low performance is caused by the fact that the jung library completely redraws the whole graph several times each second without actual reason for doing so (It doesn't only redraw the graph by displaying precomputed bitmap, it actually determines the graph appearence from scretch by calling all its transformers and renderers, which is really computationaly expensive). By extending the standard jung VisualizationViewer class and overriding few functions, that resolved the issue of unnecesarry graph redrawing, we solved the problem of the low gui performance. We don't grant that there is no easy solution for the described problem, but the fact that we couldn't actually find it gave us the notion, that there are few weak points in the design of the library.

\subsubsection {Querying Functionalities}
\label{sec:querying_functionalities}
The nodes of the graph are generated from the results of a database query supplied by the user, who can search either by keyword, by NE, or by a combination of both in the same query. Likewise, the user may select a range of years to search in (e.g.\ retrieve documents occurring between \code{1970} and \code{1975}), as well as select only certain types of documents to be searched (e.g.\ retrieve only documents of type \meta{Speech} or \meta{Interview}). The documents most relevant to the current query are displayed, the number of the documents displayed being specified by the user (e.g.\ display the \code{25} most relevant documents), and edges connecting each document are drawn based on the similarity measures relating each document to each other.

\subsubsection {Presenting Results}
\label{sec:presenting_results}
After submitting a query, a table of query results is displayed and a graph corresponding to the results is drawn, in which the relevance measure of each document to the current query is represented by the absolute size of the node representing the document in question: The larger the node, the more relevant it is to the current query. The type of document represented by the node is represented by its shape, e.g.\ \meta{Speech} $\rightarrow$ star, \meta{Meeting} $\rightarrow$ pentagon. The edges linking the nodes of the graph represent the strength of similarity the of the two documents represented by the nodes.

	
\subsection {Combining It All: User Interface} 
\label{sec:combining_it_all:_user_interface}
The query functions, graph and display functions are integrated into a single graphical front-end relying primarily on the mouse for user input.

\begin{figure}[h]
\centering
\caption{GUI}
%\includegraphics[width=160mm]{gui.png}
\end{figure}

\subsubsection{Search function}
\begin{figure}[h]
\centering
\caption{GUI}
%\includegraphics[width=160mm]{search.png}
\end{figure}

Search terms are entered in \note{picture ref}: Entering any term will return both NEs and also keywords sharing the exact form (not case-sensitive)--- For instance, entering \lingform{committee} will both return exact instances of \code{committee} in the document itself, and will return NEs similar to \lingform{committee} based on the string kernel similarity measure, e.g.\ \lingform{Central Committee} and \lingform{Central Committee of the Cuban Communist Party}. Multi-word searches are denoted in double quotes, e.g.\ \code{``Central Committee''}. The maximum amount of query results presented are specified by \note{picture ref}: For example, entering a value of \code{10} will return ten documents most relevant to the query. The set of years searched between is specified by \note{picture ref}. The type(s) of documents searched (e.g.\ \code{Speech}, \code{Interview}) is specified in \note{picture ref}. Once the criteria are entered, the query is run by \note{picture ref}.

\subsubsection{Results}
\begin{figure}[h]
\centering
\caption{GUI}
%\includegraphics[width=160mm]{representation.png}
\end{figure}

\paragraph{Presentation}
The results of the current query are displayed both in a table of results (\note{picture ref}) and in a graph displayed in the main window (\note{picture ref}).

In the table, the documents are displayed in rows sorted by their relevance to the query in descending order. The metadata associated with each document is displayed in columns.

\subsubsection{Results}
\begin{figure}[h]
\centering
\caption{GUI}
%\includegraphics[width=160mm]{nodecloseup.png}
\end{figure}

In the graph, each document is represented by a node \note{picture ref}. The relevance of the document to the query is represented by the size of the node: The larger the node, the more relevant the document is to the submitted query. The edges between the nodes represent similarities between the documents based on a particular NE the documents have in common, or two similar NEs based on the string kernel measure. The stronger the similarity connection, the heavier the line weight of the edge drawn. There are three line weights: ``thick'' \note{picture ref}, ``normal'' \note{picture ref} and ``dotted'' \note{picture ref}, in descending order of the level of similarity they represent.

\paragraph{Navigation}
Selecting either an entry in the table or a node in the graph with the mouse or keyboard displays the NEs in the document in a sidebar. The NEs in a particular document and their string kernel-based aliases are displayed in \note{picture ref}: Boldface entries represent NEs in the given document, while the non-boldface entries represent NEs similar to the boldface NEs in the document according to the string kernel similarity measure. For example, the NE \textbf{\lingform{Castro}} may also return the similar NEs \lingform{Fidel}, \lingform{Fidel Castro} and \lingform{Dr. Fidel}. The type of the NE (e.g.\ \code{Person} or \code{Organization}) is represented by the color of the text, e.g.\ red for \meta{Persons}, green for \meta{Organizations} and blue for \meta{Locations}. It also is possible to select multiple documents, e.g.\ by holding SHIFT and clicking on multiple nodes/table entries, or by clicking and dragging the mouse to create a selection box encompassing the nodes/table entries to be selected. When multiple documents are selected, the entries displayed in \note{picture ref} are those that are common to all selected documents.

\begin{figure}[h]
\centering
\caption{GUI}
%\includegraphics[width=160mm]{nodeselect.png}
\end{figure}

In addition to viewing document metadata in the query results table, the user may to view the metadata of the document associated with a particular node (e.g.\ document \meta{Type}, \meta{Date} or \meta{Location}) through the node's context menu (i.e.\ by right-clicking on the node with the mouse on \note{picture ref}). Also available through the context menu is the text of the document itself (\note{picture ref}), which is displayed in a separate window which can be kept open while still navigating through the graph and viewing other nodes. The NEs in the text are denoted by being colored according to their type. Lastly, it is possible to automatically select the nearest neighbors of the node (\note{picture ref}), that is, to mark the nodes which share a named entity directly with the said node.

\note detailed description of all the features 
