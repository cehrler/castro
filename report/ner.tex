\note{identifying names, locations and organizations in the Castro documents,
input/output examples, evaluation??}

We chose the Stanford Named Entity Recognizer for the task of identifying named entities in the collection.
The Stanford NER is considered to be one of the best freely available tools for named entity recognition.
It recognizes location, persons and organizations which, as mentioned previously, are all extremely relevant for the historical domain. 
It comes with models pre-trained on the CoNLL, MUC-6, MUC-7 and ACE named entity corpora. As a result, it is quite robust across domains.

The pre-trained modules are very valuable for us, as we do not have NE annotated data for the Castro collection, and manual annotation of NE's is a labor intensive task. As discussed in the evaluation section \ref{sec:named_entity_recognition}, the domain robustness of the tool is reflected in the quality of the NE tagging for our collection, which contains a variety of different genres. 

The Stanford Named Entity Recognizer implements linear chain Conditional Random Field sequence models, first proposed by \cite{lafferty2001conditional}.
CRF is a discriminative model for finding the most probable sequence of labels (NE types) given a sequence of observations (words). 
It does not assume independence between features, and enables conditioning each random variable of the complete input sequence. The features used for the recognizer implementation include word features, orthographic features, label sequences, ID's of word clusters computed according to distributional similarity. 

CARSTEN - PROVIDE STATISTICS TABLE, WRITE ON TECHNICALITIES WITH RUNNING THE NER AND STORING RESULTS



