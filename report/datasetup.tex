\note{reading the data, creating MySQL database, creating data model}

At a first step we used a \texttt{shell}\footnote{http://tiswww.case.edu/php/chet/bash/bashtop.html}
script and \texttt{wget}\footnote{http://www.gnu.org/software/wget/} to download the complete castro
speech database. The actual speech data consists of a set of static webpages sorted by year into
folders. In a second processing step we parsed these html-files and extracted the actual content of
each file and its meta-information into textual format. To this end, several \texttt{shell} and
\texttt{python}\footnote{http://www.python.org/} scripts have been written.

The extracted information serves as the basis for the consecutive analysis and processing steps. In
order to provide structured access to the meta-information we developed a
\texttt{perl}\footnote{http://www.perl.org/} script to convert the meta-information for each
document into an entry in a \texttt{mysql}\footnote{http://www.mysql.de/products/enterprise/}
database. For every document, the database provides the following information
\begin{itemize}
 \item{\textsc{Author:} Name of the author of the document.}
 \item{\textsc{Location:} The place the document orginated.}
 \item{\textsc{Headline:} The headline of the document.}
 \item{\textsc{Date:} The date the document orginated.}
 \item{\textsc{ReportDate:} The date th3e content of the document appeared in written from.}
 \item{\textsc{Type:} There are several different types of documents in the databse. E.g.
 \texttt{SPEECH}, \texttt{INTERVIEW}, \texttt{ARTICLE}, etc.}
 \item{\textsc{Header:} Additional meta-information}
 \item{\textsc{Source:} The source where this document appeared in written form the first time.}
\end{itemize}
In addition to this information, we later enriched the database with the extracted named entities
(see \ref{sec:named_entity_recognition}) for each document. The columns \textsc{Persons},
\textsc{Places}, \textsc{Organizations} have been added and each field contains the bulk information
of the identified named entities as a comma separated list. Although this representation is is not
the most elegant and optimal solution, the derived database provides sufficient performance and
usability for the task at hand.

\subsection {Named Entity Recognition}
\label{sec:named_entity_recognition}
Please recall, that our ultimate goal is to come up with a cross-document recommendation system for
the castro speech database. Therefore, we need an appropriate metric on between document
similarity that serves as a basis for the recommendation software.

Our first idea was to use proper \textit{co-reference resolution}, e.g. provided by
\texttt{BART}\footnote{http://www.bart-coref.org/} software \cite{bart}. In this case, document
similarity is a function of the number of co-references that appear between two docuemnts. However,
in the end we decided to drop this approach because of the overall complexity and the tight time
constraints. In general, between document co-reference resolution far from an easy task. The castro
database though has some nice properties that allow for a more straightforward solution. For this
dataset, the domain is rather limited and the number of distinct topics should be small. Hence, we
came up with the idea to base the similarity measure on smoothed co-occurences of named entities.

The next step in the pre-processing chain is thus the execution of named entity recognition followed
by a smooting of the features in the vector space induced by the named entities using kernel
methods (see \ref{sec:string_kernel}) \cite{string_kernel_coref}.