At a first step we used a \texttt{shell}\footnote{http://tiswww.case.edu/php/chet/bash/bashtop.html}
script and \texttt{wget}\footnote{http://www.gnu.org/software/wget/} to download the complete castro
speech database. The actual speech data consists of a set of static web-pages sorted by year into
folders. In a second processing step we parsed these html-files and extracted the actual content of
each file and its meta-information into textual format. To this end, several \texttt{shell} and
\texttt{python}\footnote{http://www.python.org/} scripts have been written.

The extracted information serves as the basis for the consecutive analysis and processing steps. In
order to provide structured access to the meta-information we developed a
\texttt{perl}\footnote{http://www.perl.org/} script to convert the meta-information for each
document into an entry in a \texttt{mysql}\footnote{http://www.mysql.de/products/enterprise/}
database. For every document, the database provides the following information
\begin{itemize}
 \item{\textsc{Author:} Name of the author of the document.}
 \item{\textsc{Location:} The place the document originated.}
 \item{\textsc{Headline:} The headline of the document.}
 \item{\textsc{Date:} The date the document originated.}
 \item{\textsc{ReportDate:} The date th3e content of the document appeared in written from.}
 \item{\textsc{Type:} There are several different types of documents in the database. E.g.
 \texttt{SPEECH}, \texttt{INTERVIEW}, \texttt{ARTICLE}, etc.}
 \item{\textsc{Header:} Additional meta-information}
 \item{\textsc{Source:} The source where this document appeared in written form the first time.}
\end{itemize}
In addition to this information, we later enriched the database with the extracted named entities
(see \ref{sec:named_entity_recognition}) for each document. The columns \textsc{Persons},
\textsc{Places}, \textsc{Organizations} have been added and each field contains the bulk information
of the identified named entities as a comma separated list. Although this representation is is not
the most elegant and optimal solution, the derived database provides sufficient performance and
usability for the task at hand.
