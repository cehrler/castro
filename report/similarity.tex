%\subsubsection{Computing Similarity between Documents}\label{sec:computing_similarity_between_documents}

As mentioned previously, our representation of the documents is based on their similarity to each other according to some extracted data. The edges of the resulting graphs are constructed based on these similarities. The crucial and difficult question is, how do the similar and related documents look like. This question can have various answers depending on the set of documents and the interests of the historian. Our working hypothesis is that the similarity and relatedness between the documents that might be relevant for the interest of the historian can be expressed as the similarity of the named entities. 
The assumption is that NE's of the sort we extract are more relevant for historically motivated similarity then other information that can be extracted from the documents. As we represent documents in VSM as the vectors of named entities, we can directly apply standard vector space measures for expressing similarities between vectors. 

The first measure we use is the cosine similarity measure. 
\[\text{cosim}(d,d') = \sum_{k \in \mathcal{T}}\frac{d(k)d'(k)}{|d||d'|}\]
It expresses the cosine angle between the vectors $d$ and $d'$. The cosine measure is equal to $1$ for the parallel vectors and is equal to $0$ for perpendicular vectors. The smaller the angle between the vectors, the more similar they are.

We also used "Manhattan similarity measure". The Manhattan similarity between two vectors can be described using the Manhattan distance between the zero vector and the minimum value for each column out of the two vectors.

\[\text{manhattan}(d,d') = \sum_{k \in \mathcal{T}}min[d(k), d'(k)]\]

We briefly compared both similarity measures by comparing the similarity graphs constructed using those measures. Though at first glance the edges look differently, it seems that the main reason for that concerns the different scalability of the two measures. Adjusting the similarity threshold of one of the graphs can yield rather similar edges to the other graph.

In the current version we use cosine similarity measure by default. We do that because this measure is more widespread that the Manhattan distance, and because of its elegant mathematical interpretation.

We computed similarity matrices in order to store the similarity rates between each pair of documents. Separate similarity matrices were computed for each combination of the four indexes, two similarity measure and two aliasing settings (aliased or non-aliased), resulting is 16 similarity matrices. We explain our approach to aliasing and its effect on the similarity measures in section \ref{sec:aliasing}. Unlike the index matrices, the similarity matrices are not sparse and each entry is represented in the memory.

Similarity matrices are loaded on the startup of the application. Switching similarity measures or aliasing settings leads to redrawing of the edges of the graph according to a different matrix. Choosing a weights combination of NE's and possibly general vocabulary leads to on-the-fly computation of similarities based on the specified linear combination of values from the relevant precomputed similarity matrices.



