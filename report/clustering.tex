For a recommendation system based on graphs it is a desirable feature to cluster the graph into
dense regions. For such regions it is assumed, that they correspond to highly similar or related
documents and such clusters can form a group of a larger related unit. Hence, being able to
automatically identify clusters can help to further discern between important and less important
documents for the user.

There exist several algorithms for clustering of graphical data, including rather simple
hierarchical clustering methods and classical methods like k-means clustering
\cite{2}. Amongst todays most popular clustering algoirthms are Spectral Clustering
methods, which are based on eigendecompositions of graph laplacians \cite{spectral_clustering}.

For the purpose of this project two clustering methods were implemented: First, $k$-means clustering
and secondly Chinese Whispering clustering which was especially developed for NLP tasks and has the
favorable property that it automatically determines the number of clusters.

\subsubsection{k-Means Clustering}
\label{sec:k-means_clustering}
At first $k$-means clustering was implemented and evaluated. For this case the VMS representation of
documents is used, i.e.\ the same vectors as in the index representation of the documents (see
Section \ref{sec:indexing}). However, one drawback of $k$-means is that the user needs to specify
the number of clusters $k$ a priori.

The objective in $k$-means clustering is to minimize the total within cluster variance. For the
initialization, $k$ cluster prototypes are randomly placed on the data. Then the following two
steps are iterated until convergence:
\begin{itemize}
  \item Assign each datapoint to its closest prototype.
  \item For each prototype compute the center of mass over all assigned datapoints and place the
prototype in the new center of mass.
\end{itemize}

This approach did not perform very well. The resulting clusters were not the densely connected
regions which is probably caused by the fact that the dimension of the vector space is very high
and that vectors representing the documents are very sparse. Hence, an other clustering algorithm
for graphical data was used.

\subsubsection{Chinese Whispering Clustering}
\label{sec:chinese_whipsering_clustering}
The Chinese Whispering (CW) clustering alorithm was developed with a focus on NLP tasks and it is a
ranomized time-linear graph clustering algorithm with the special feature that it does not depend
on external parameters but adapts to the domain \cite{cw_clustering}.

In the initialization each node is assigned to its own cluster. Then the follwoing update round is
perferomed until the assignment of classes to nodes does not change anymore: In a randomized
fashion, each node is assigned to the currently highest ranked class in the local neighborhood.
After few iterations the stable clustering is achieved (i.e.\ there are no changes in the cluster
assignments) or there are few nodes at the border of the clusters which periodically flip their
cluster assignment.

Although this is a very basic clustering alorithm, the performance on our data was satisfactory. The
algorithm performed much better than k-means clustering. However, one of the drawbacks of CW is due
to that teh algorithms always assigns nodes to the locally dominant cluster, although sometimes the
node should rather stay in his own cluster.

To handle this problem we implemented a variation of this algorithm. In the modified implementation
the cluster assignment sum needs to be greater than a predefined threshold, otherwise the current
node does not change its cluster assignment. In the current implementation the threshold is set to
the average edge strength of the whole graph multiplied by the constant specified by the user.

Before the actual update rounds start, which works exactly like the update rounds in the standard
version with the exception of the thresholding feature described in previous paragraph, are started
the predefined number of initialization steps are done.

In the initialization step, all edges of the graph are sorted in the descending order according to
their weights. In each initialization step, one each is pulled out from the list and the standard
node cluster assignment procedure is then executed on the node, for which the cluster assignment sum
is greater. It means that the cluster assignment sum doesn't need to exceed the precomputed
threshold.

By this we want to start up the clustering process because it may be hard to exceed the precomputed threshold at any node when all the nodes have their own cluster. Selecting the nodes attached to the edges with great weight ensures that the cluster assignments done during these initialization steps are not relevant.

The modified Chinese Whisper algorithm is a recent feature and it was not properly evaluated yet.
Visual inspections with reasonable parameter settings revealed, that the the resulting cluster
assignment look better, compared to the ones obtained from classic CW clustering. However, for the
clustering algorithm there is plenty of room for further improvement and a need for extended
evaluation. Ideas for improvement are e.g.\ experimentation with the thresholding constant
and implementation of different initialization schemas.

To summarize, graph clustering has the potential enhance our recommendation system in such a way
that it can help the user to identify tightly coupled sets of documents in larger graphs that
can be subject of further investigation.

