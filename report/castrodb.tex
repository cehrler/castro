\paragraph{Structure}
The \citeA{CastroDB} is an online corpus consisting of speeches, interviews, press conferences and other oratory by Fidel Castro from 1959 to 1996 translated into English, maintained by the Latin American Network Information Center (LANIC) at the University of Texas at Austin. The collection contains 1492 documents with an average document size of approximately 3,800 words. The information is presented as plain-text documents embedded into HTML web pages organized by the year and month of the original release. The documents are annotated with metadata including:

\begin{itemize}
\item The type of document (e.g.\ \scare{speech}, \scare{interview}, \scare{press conference})
\item The exact date of the original release
\item The author, e.g.\ Castro himself or a particular reporter
\item The original source, e.g.\ \scare{Havana Domestic Radio}
\item A headline (such as in a newspaper)
\item The reporting agency, e.g.\ the Foreign Broadcast Information Service (FBIS)
\item The exact date of the report
\end{itemize}

\paragraph{Linguistic properties}
The documents are manually translated from modern Spanish into modern English, sparing us many problems associated with documents written in historical languages. This characteristic, along with a considerable amount of newspaper comparable content allow a relatively straightforward application of technologies already deployed and currently used in NLP and IR.

However, the corpus is very heterogeneous: Not only does it span 37 years of oratory, but it also incorporates many different authors. Although it is a database dedicated to the oratory of Fidel Castro, it features many documents which are not narrated by Castro, for instance by news reporters, journalists and interviewers. Likewise, the corpus comprises many different genres and styles.

Secondly, many of the documents in the corpus are (at least indirectly) based on spoken language and so are not structured in a formal way, in polar opposition to e.g.\ a scientific report. Likewise, many documents in the corpus feature heavily-rhetorical content. Due to this characteristic, in many documents, topics are multiple and often vague. This observation is valid for the whole documents as well as smaller sections such as single paragraphs. 

In this way, the corpus is less similar to relatively-structured corpora (such as the Wall Street Journal-based \citeauthor{PTB}) it is to less-structured and more heterogeneous literary corpora. On these kinds of documents, topic identification as well as document summarization and clustering is difficult to implement manually (using human competence), let alone doing so automatically.
