\paragraph{Structure}
The \citeA{CastroDB} (CSDB) is an online corpus consisting of speeches, interviews and press conferences and other oratory by Fidel Castro from 1959 to 1996 translated into English, maintained by the Latin American Network Information Center (LANIC) at the University of Texas at Austin. The information is presented as plain-text documents embedded into HTML web pages organized by the year and month of the original release. The documents are annotated with metadata including:

\begin{itemize}
\item The type of document (e.g.\ \scare{speech}, \scare{interview}, \scare{press conference})
\item The exact date of the original release
\item The author, e.g.\ Castro himself or a particular reporter
\item The original source, e.g.\ \scare{Havana Domestic Radio}
\item A headline (such as in a newspaper)
\item The reporting agency, e.g.\ the Foreign Broadcast Information Service (FBIS)
\item The exact date of the report
\end{itemize}

\paragraph{Linguistic properties}
From a technical perspective, the task of IR is facilitated by the fact that documents are translated into modern English, allowing the usage of modern technologies already deployed and currently used in IR.

However, the corpus is very heterogeneous: Not only does it span 37 years, but it also incorporates many different authors--- Although it is a database dedicated to the oratory of Fidel Castro, it features many documents which are not narrated by Castro, for instance by news reporters, journalists and interviewers. Likewise, the corpus comprises many different genres and styles.

Searching through the corpus is a lengthy and laborious process: Firstly, the organization of the corpus is one-dimensional (solely by date). Secondly--- and possibly more importantly--- many of the documents in the corpus are (at least indirectly) based on spoken language and are thus not structured in a formal way--- in polar opposition to e.g.\ a scientific report. Likewise, many documents in the corpus feature heavily-rhetorical content and may in fact cover multiple topics in one document. Due to this characteristic, topic identification and document summarization and clustering is difficult to implement--- either automatically or manually, by someone interested in obtaining information from the corpus.