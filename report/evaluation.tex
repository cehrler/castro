\note{FROM CAROLINE: Evaluation 
of sub-components and system overall (insofar as it makes sense) 
either quantitative (e.g., accuracy for information extraction 
task) and/or qualitative (e.g., give examples where your system 
improves IR over keyword-based search) 
also discuss cases where your system doesn't perform well and 
describe how it could be improved (error analysis)}

\subsection {Named Entity Recognition}
\label{sec:named_entity_recognition}
While the Stanford CRF-based NER system is highly accurate and recognizes the majority of NEs in any given document, it does not always correctly recognize the full and exact substring corresponding to the NE in question. For example, the location \lingform{Santiago de Cuba} was often recognized as e.g.\ \lingform{Santiago} \lingform{Santiago de} or \lingform{Santiago de Cuba-is}, and may even be recognized as two NEs--- as \lingform{Santiago} and \lingform{Cuba}:

Similarly, the type of NE recognized may be incorrect. For example, \lingform{Lenin} was recognized as \meta{Location} on more than one occasion. These problems can also occur concurrently: Just as with \lingform{Santiago de Cuba}, \lingform{Fidel Castro} may not be recognized in its entirety as a single NE, and more than once was recognized as two separate NEs of different types: \lingform{Fidel} as \meta{Organization} and \lingform{Castro} as \meta{Person}.

Although the total error rate is small, it is not inconsequential: For example, in one document\footnote{ID 396: \scare{Speech to Textile Workers in Tome}}, there were 37 NEs. Four of the NEs were not recognized at all, one was only recognized partially, and one was recognized as the wrong type. It seems that while the NER is mostly accurate, the errors made in NER are propagated to all downstream components, affecting the end results.


\subsection {String Kernels and Co-reference Resolution}
\label{sec:string_kernels_and_co-reference_resolution}
As described in section \ref{sec:aliasing} string kernels allow greater flexibility than other common string similarity measures such as edit distance.
For example, the following organizations are identified as aliases of each other:
\begin{enumerate}
\item  \lingform{Public Health Ministry}
\item  \lingform{PublicHealth Ministry}
\item  \lingform{Public HealthMinistry}
\item  \lingform{Ministryof Public Health}
\item  \lingform{Ministry ofPublic Health}
\item  \lingform{Ministry of Public Health}
\item  \lingform{Ministry of PublicHealth}
\end{enumerate}
All the items in this group seem to refer to the same extra-linguistic object. Items 1--3 are very close to each other in terms of edit distance.
The same can also be said about items 4--7. String kernels not only account for these similarities, but also associate between the different word arrangements of the expression. Using edit distance would not be able to bind such cases. A related example is the location \lingform{Delhi} and its aliases 
\lingform{NewDelhi} and \lingform{New Delhi}. Here, the additional value of the string kernel method comes in the form of associating the shorter version of the
name with its longer variant. 

However, in many cases, the flexibility of string kernels results in associating named entities which are entirely unrelated to each other. For example: \lingform{Polish People's Republic}, \lingform{Lao People's Republic}, \lingform{Mongolian People's Republic}

This is particularly problematic with names of people. For instance, there are 21 different people
whose last name is \lingform{Rodriguez} (or a variant of this name) and who were wrongly aliased
together. Another case of common mistakes are long titles such as \lingform{Comrade} which create
rather large clusters of unrelated names.   
 
Naturally, along with the cases of clear success and those of clear failure, many aliases contain a mixture of correct and incorrect bindings, as well as bindings whose correctness can only be determined in context--- consider:

\begin{enumerate}
 \item \lingform{Raul Castro}
 \item \lingform{MajRaul Castro}
 \item \lingform{GenRaul Castro}
 \item \lingform{RaulCastro}
 \item \lingform{Mr Castro}
 \item \lingform{F. Castro}
 \item \lingform{Dr Castro}
 \item \lingform{RaulCastro Ruz}
 \item \lingform{Maj RaulCastro}
 \item \lingform{Gen RaulCastro}
 \item \lingform{Raul Castro Ruz}
 \item \lingform{Maj Raul Castro}
 \item \lingform{Gen Raul Castro}
\end{enumerate}

Most of the strings above refer to the same entity (Raul Castro), with the exception of \lingform{F. Castro}, which clearly refers to a different entity, as well as \lingform{Mr Castro} and \lingform{Dr Castro} (the correctness of these forms being dependent on context).

By examining the outputs of using different similarity thresholds, it was found that the original assumption that the trade-off between recall and precision can be partially regulated by adjusting the similarity threshold. A higher threshold leads to higher precision but lower recall, while a lower threshold improves recall at the expense of precision.

However, even given this setup, the mechanism has considerable over-generation, which then leads to unjustified linking between documents which is a crucial problem. The noise produced by the aliasing over-generation is possibly the major drawback of our system. We discuss solutions for this situation in the future research section \ref{sec:future_work_NLP}.




\subsection {Document Similarity}
\label{sec:document_similarity}

\subsection {Information Retrieval}
\label{sec:information_retrieval}

\subsection {GUI Performance}
\label{sec:gui_performance}
The results of the document similarity visualization are mixed: While there are many relationships represented between documents, the qualitative value of these relationships is irregular.

For example, a query including the named entity \code{"Central Committe"} and the keyword \code{economy} for documents between the years of 1959 and 1984 produced a cluster of documents including the following headlines:

\begin{enumerate}
\item ``USSR--Cuban Fishing Base Pact Signed''\label{eval:doc179}
\item ``Fidel Castro sends regrets to Dominican Olympic Committee''\label{eval:doc553}
\item A meeting of Fidel Castro with Salim Rabay'l `Ali, assistant secretary of the Central Committee of the United Political Organization National Front\label{eval:doc682}
\item A meeting of Fidel Castro with Jesse Jackson regarding US--Cuban normalization\label{eval:doc879}
\item A meeting of Fidel Castro with Mrs. Nguyen Thi Binh, minister of foreign affairs of the Republic of South Vietnam\label{eval:doc643}
\item An address by Fidel Castro to the people of Chile regarding Chilean--Cuban relations and working for a better future\label{eval:doc391}
\item Fidel Castro visits the South Vietnamese embassy in Cuba to welcome ``the great victory \ldots in liberating Saigon'' and inquire about social, economic and political conditions in South Vietnam\label{eval:doc626}
\end{enumerate} 

This cluster was generated with a relative edge density of 2.52 (relatively high) while also applying a vertex distance filter of 1, selecting document \ref{eval:doc179} as a focal point. In other words, the graph was set to display many relations between all documents which shared at least one named entity with the document in question. Of the six documents connected to this document in the resulting graph (shown above), only one--- \ref{eval:doc626}--- is qualitatively related to document \ref{eval:doc179} in a direct sense.

However, all these documents contain the named entity \lingform{Central Committee} or a named entity similar to it according to the string kernel similarity measure. This is often realised as \lingform{First secretary of the \textbf{Central Committee} of the Communist Party of Cuba} or a variant of it--- which often co-refers to \lingform{Fidel Castro} or a variant of it.

It seems then that the document clustering is more relevant than first observed: All the documents contain references to \code{Fidel Castro}. However, this is still of little practical use, as the named entity \code{Fidel Castro} is in the vast majority of the documents in the corpus: This clustering does not distinguish a document or group of documents in a way which is of great use to a user browsing the corpus.

However, querying more specific terms revealed different results: Querying \code{"Giron"}, and \code{"Kennedy"} produced a graph of documents which are much more similar to each other in a qualitative sense than that produced by the previous query, for example:

\begin{enumerate}
\item ``Castro marks 5th anniversary of [Bay of Pigs] invasion'' \label{eval:doc274}
\item Speech for funeral of five Cuban troops killed in Baracoa landing operation [of the Bay of Pigs invasion]\label{eval:doc337}
\item ``Castro Remarks at Missile Crisis Conference\label{eval:doc1418}
\item ``Castro marks 2nd anniversary of [Bay of Pigs]'' invasion \label{eval:doc210}
\item An article on US--Cuban--Soviet relations, including consequences of ``Giron'' (the Bay of Pigs) and the Cuban Missile Crisis, mentioning President Kennedy\label{eval:doc864}
\item Report of the arrival of Algerian Preimier Ben Bella in Havana and Castro's speech for it--- in which he mentions the ``victory'' at ``Giron''
\item ``Castro discusses Latin America in EFE interview'', mentioning Kennedy and several of his actions regarding Latin America \label{eval:doc923}
\item ``Castro denounces U.S. aggression'' (April 1961) \label{eval:doc134} (regarding 
\end{enumerate} 

This cluster was generated with a relative edge density of 2.52 with a vertex distance filter of 1 selecting document \ref{eval:doc134} as a focal point. Of the seven total documents in cluster, all but one of them are directly related to the Bay of Pigs invasion itself, US aggression and/or President Kennedy. Even more interesting is that the document \ref{eval:doc337} was relatively isolated in the graph compared to the many edges connecting the anniversary speeches and politically-motivated documents such as \ref{eval:doc923} and \ref{eval:doc134}.

However, such results were obtained only with relatively specific named entities--- Contrast the query involving \code{``Giron''} and \code{``Kennedy''} to that involving \code{``Central Committee''} ($\sim$ \code{``Castro''})  and \code{economy}. Additionally, even after entering a relatively specific query, much adjustment of the visualization settings was required to display an graph which was both of practical use to the user and was not overwhelming in size and complexity.