\note{FROM CAROLINE: Evaluation 
of sub-components and system overall (insofar as it makes sense) 
either quantitative (e.g., accuracy for information extraction 
task) and/or qualitative (e.g., give examples where your system 
improves IR over keyword-based search) 
also discuss cases where your system doesn't perform well and 
describe how it could be improved (error analysis)}

\subsection {Named Entity Recognition}
\label{sec:named_entity_recognition}
While the Stanford CRF-based NER system is highly accurate and recognizes the majority of named entities in any given document, it does not always correctly recognize the full and exact substring corresponding to the named entity in question: It is possible that a substring of the substring corresponding to the named entity is not recognized, and it is also possible that words outside the substring denoting the named entity in question may be falsely recognized as part of the said substring:

\begin{table}
\centering
\caption{NE form errors}
\begin{tabular}{l r | l r }\toprule
 \textbf{Form} & \textbf{Relative frequency} & \textbf{Form} & \textbf{Relative frequency} \\
\midrule
\emph{Santiago de Cuba} & & & \\
\midrule
Santiagode Cuba & & & \\
Santiago deCuba & & & \\
SantiagodeCuba & & &\\
Santiago de Cuba-is & & &\\
\bottomrule
\end{tabular}
\end{table}

\begin{table}
\centering
\caption{NE form errors}
\begin{tabular}{l r | l r }\toprule
 \textbf{Form} & \textbf{Relative frequency} & \textbf{Relative frequency} \\
\midrule
\multicolumn{2}{l}{\emph{Cuban}} & \multicolumn{2}{r}{}\\
\multicolumn{2}{l}{\emph{Soviet}} & \multicolumn{2}{r}{}\\
\multicolumn{2}{l}{\emph{French}} & \multicolumn{2}{r}{}\\
\midrule
Cuban-Soviet & & Cuban-French & \\
Soviet-Cuban & & Soviet-French & \\
French-Cuban & & French-Soviet & \\
\bottomrule
\end{tabular}
\end{table}

\subsection {String Kernels and Co-reference Resolution}
\label{sec:string_kernels_and_co-reference_resolution}
As mentioned in the introduction, true CR in a traditional sense was foregone in the beginning of the project in favour of a string kernel-based approach \note{CITATION NEEDED}. Similarly to CRR, the string kernel-based similarity measure allowed many permutations of a similar named entity to be correlated (e.g.\ \textbf{\lingform{Castro}} $\rightarrow$ \{\lingform{Fidel Castro}, \lingform{Dr. Fidel}, \lingform{DrFidel} \ldots\}). However, the the string kernel method as it is currently implemented overgenerates such similarities--- for example, \textbf{\lingform{State Technical University}} $\rightarrow$ \{\lingform{Socialist Youth}, \lingform{API}, \lingform{Socialist Party}, \lingform{State Committee}, \lingform{forTechnical}, \lingform{Catholicism} \ldots\}.

In effect, the string kernel method often associates unrelated NEs, unrelated documents are erroneously related with each other, causing recall to be relatively high in comparison to the precision obtained through the method. This in turn causes a large amount of noise in the graph, making it harder for the user to discern relationships between documents which may be useful to the user.

\subsection {Document Similarity}
\label{sec:document_similarity}

\subsection {Information Retrieval}
\label{sec:information_retrieval}

\subsection {GUI Performance}
\label{sec:gui_performance}
The results of the document similarity visualization are mixed: While there are many relationships represented between documents, the qualitative value of these relationships is irregular.

For example, a query including the named entity \code{"Central Committe"} and the keyword \code{economy} for documents between the years of 1959 and 1984 produced a cluster of documents including the following headlines:

\begin{enumerate}
\item ``USSR--Cuban Fishing Base Pact Signed''\label{eval:doc179}
\item ``Fidel Castro sends regrets to Dominican Olympic Committee''\label{eval:doc553}
\item A meeting of Fidel Castro with Salim Rabay'l `Ali, assistant secretary of the Central Committee of the United Political Organization National Front\label{eval:doc682}
\item A meeting of Fidel Castro with Jesse Jackson regarding US--Cuban normalization\label{eval:doc879}
\item A meeting of Fidel Castro with Mrs. Nguyen Thi Binh, minister of foreign affairs of the Republic of South Vietnam\label{eval:doc643}
\item An address by Fidel Castro to the people of Chile regarding Chilean--Cuban relations and working for a better future\label{eval:doc391}
\item Fidel Castro visits the South Vietnamese embassy in Cuba to welcome ``the great victory \ldots in liberating Saigon'' and inquire about social, economic and political conditions in South Vietnam\label{eval:doc626}
\end{enumerate} 

This cluster was generated with a relative edge density of 2.52 (relatively high) while also applying a vertex distance filter of 1, selecting document \ref{eval:doc179} as a focal point. In other words, the graph was set to display many relations between all documents which shared at least one named entity with the document in question. Of the six documents connected to this document in the resulting graph (shown above), only one--- \ref{eval:doc626}--- is qualitatively related to document \ref{eval:doc179} in a direct sense.

However, all these documents contain the named entity \lingform{Central Committee} or a named entity similar to it according to the string kernel similarity measure. This is often realised as \lingform{First secretary of the \textbf{Central Committee} of the Communist Party of Cuba} or a variant of it--- which often co-refers to \lingform{Fidel Castro} or a variant of it.

It seems then that the document clustering is more relevant than first observed: All the documents contain references to \code{Fidel Castro}. However, this is still of little practical use, as the named entity \code{Fidel Castro} is in the vast majority of the documents in the corpus: This clustering does not distinguish a document or group of documents in a way which is of great use to a user browsing the corpus.

However, querying more specific terms revealed different results: Querying \code{"Giron"}, and \code{"Kennedy"} produced a graph of documents which are much more similar to each other in a qualitative sense than that produced by the previous query, for example:

\begin{enumerate}
\item ``Castro marks 5th anniversary of [Bay of Pigs] invasion'' \label{eval:doc274}
\item Speech for funeral of five Cuban troops killed in Baracoa landing operation [of the Bay of Pigs invasion]\label{eval:doc337}
\item ``Castro Remarks at Missile Crisis Conference\label{eval:doc1418}
\item ``Castro marks 2nd anniversary of [Bay of Pigs]'' invasion \label{eval:doc210}
\item An article on US--Cuban--Soviet relations, including consequences of ``Giron'' (the Bay of Pigs) and the Cuban Missile Crisis, mentioning President Kennedy\label{eval:doc864}
\item Report of the arrival of Algerian Preimier Ben Bella in Havana and Castro's speech for it--- in which he mentions the ``victory'' at ``Giron''
\item ``Castro discusses Latin America in EFE interview'', mentioning Kennedy and several of his actions regarding Latin America \label{eval:doc923}
\item ``Castro denounces U.S. aggression'' (April 1961) \label{eval:doc134} (regarding 
\end{enumerate} 

This cluster was generated with a relative edge density of 2.52 with a vertex distance filter of 1 selecting document \ref{eval:doc134} as a focal point. Of the seven total documents in cluster, all but one of them are directly related to the Bay of Pigs invasion itself, US aggression and/or President Kennedy. Even more interesting is that the document \ref{eval:doc337} was relatively isolated in the graph compared to the many edges connecting the anniversary speeches and politically-motivated documents such as \ref{eval:doc923} and \ref{eval:doc134}.

However, such results were obtained only with relatively specific named entities--- Contrast the query involving \code{``Giron''} and \code{``Kennedy''} to that involving \code{``Central Committee''} ($\sim$ \code{``Castro''})  and \code{economy}. Additionally, even after entering a relatively specific query, much adjustment of the visualization settings was required to display an graph which was both of practical use to the user and was not overwhelming in size and complexity.