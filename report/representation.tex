\note{MICHAL - EXPAND SECTION WITH STUFF ABOUT THE COMPUTATIONAL, ALGORITHMIC, AND IMPLEMENTATION STUFF.}

Using the similarity matrix generated, a graph of document similarities is generated: Each document is represented as a node, which is connected to other documents in the graph by edges representing the similarity measure of the two documents.

\subsubsection {Querying Functionalities}
\label{sec:querying_functionalities}
The nodes of the graph are generated from the results of a database query supplied by the user, who can search either by keyword, by NE, or by a combination of both in the same query. Likewise, the user may select a range of years to search in (e.g.\ retrieve documents occurring between \code{1970} and \code{1975}), as well as select only certain types of documents to be searched (e.g.\ retrieve only documents of type \meta{Speech} or \meta{Interview}). The documents most relevant to the current query are displayed, the number of the documents displayed being specified by the user (e.g.\ display the \code{25} most relevant documents), and edges connecting each document are drawn based on the similarity measures relating each document to each other.

\subsubsection {Presenting Results}
\label{sec:presenting_results}
After submitting a query, a table of query results is displayed and a graph corresponding to the results is drawn, in which the relevance measure of each document to the current query is represented by the absolute size of the node representing the document in question: The larger the node, the more relevant it is to the current query. The type of document represented by the node is represented by its shape, e.g.\ \meta{Speech} $\rightarrow$ star, \meta{Meeting} $\rightarrow$ pentagon. The edges linking the nodes of the graph represent the strength of similarity the of the two documents represented by the nodes.