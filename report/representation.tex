\note{MICHAL - EXPAND SECTION WITH STUFF ABOUT THE COMPUTATIONAL, ALGORITHMIC, AND IMPLEMENTATION STUFF.}

So far we have described how the nodes and edges in the graph are obtained. In order to display the graph, a graph layout (two dimensional representation of the graph) must be determined. If our vector representation of the document was only two dimensional then it would be very straightforward to obtain the graph layout
by just using the two feature values as x and y coordinates. However the document in the described model is represented by the vector of several thousands of dimensions. This problem is usually not solved by direct mapping from high dimensional vector space to the low dimensional vector space. There are algorithm that solve the problem of graph layouting directly by using only the properties of the underlying graph (nodes and edges together with their weights).

The layouting algorithm that was used in our project belongs to the popular class of the layouting algorithms called force-based layout algorithms. In this class of algorithms physical metaphor is used. Spring is considered to be string and the node is considered to be electrically charged particle (all the nodes have the same charge). The optimal length of the edge/spring is set in advance. if the length of the edge exceeds the optimal value, then there is a power between connected nodes trying to push them together. If the distance of two nodes is lower than the predefined

Using the similarity matrix generated, a graph of document similarities is generated: Each document is represented as a node, which is connected to other documents in the graph by edges representing the similarity measure of the two documents.

\subsubsection {Querying Functionalities}
\label{sec:querying_functionalities}
The nodes of the graph are generated from the results of a database query supplied by the user, who can search either by keyword, by NE, or by a combination of both in the same query. Likewise, the user may select a range of years to search in (e.g.\ retrieve documents occurring between \code{1970} and \code{1975}), as well as select only certain types of documents to be searched (e.g.\ retrieve only documents of type \meta{Speech} or \meta{Interview}). The documents most relevant to the current query are displayed, the number of the documents displayed being specified by the user (e.g.\ display the \code{25} most relevant documents), and edges connecting each document are drawn based on the similarity measures relating each document to each other.

\subsubsection {Presenting Results}
\label{sec:presenting_results}
After submitting a query, a table of query results is displayed and a graph corresponding to the results is drawn, in which the relevance measure of each document to the current query is represented by the absolute size of the node representing the document in question: The larger the node, the more relevant it is to the current query. The type of document represented by the node is represented by its shape, e.g.\ \meta{Speech} $\rightarrow$ star, \meta{Meeting} $\rightarrow$ pentagon. The edges linking the nodes of the graph represent the strength of similarity the of the two documents represented by the nodes.
