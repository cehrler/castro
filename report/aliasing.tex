In general, the idea of kernel methods is to map objects of a
distinct type $K$ into some \textit{Hilbert space} $\mathcal{H}$ using a mapping function $\Phi :
K \rightarrow \mathcal{H}$. Here, $K$ can have an arbitrary structure, e.g. trees, strings,
graphs, etc. Given such a mapping $\Phi$ the similarity of two objects $s,t \in K$ can be computed
thorugh the ordinary scalar product defined on the Hilbert space:
\[sim(s,t) = <\Phi(s),\Phi(t)>\]
Instead of defining the mapping function $\Phi$ explicitly, a kernel function $k : K \times K
\rightarrow \mathbb{R}$ can be used. Kernel functions are positive-semidefinite functions
who implicitly compute the scalar product of two objects on some Hilbert space. That is, for every
kernel function $k$ there exists a mapping $\Phi$, such that
\[k(s,t) = <\Phi(s),\Phi(t)>.\]
This property of kernel functions is often exploited in practice because it greatly facilitates the
computational cost that is needed to compute object similarity.

\textit{String kernels} are kernel methods \note{cite} that operate on the domain of strings over
a finite alphabet $\Sigma$ and are often applied in the fields of \textit{information
retrieval} and \textit{bioinformatics}. For the purpose of our work, we applied a
\textit{p-spectrum} string kernel \note{citation needed} which is defined by
\[k_p(s,t) = \sum_{u \in \Sigma^p}{\phi_u^p(s)\phi_u^p(t)},\]
where $\phi_u^p(s)$ counts the number of occurrences of the substring $u$ of length $p$ in $s$. The
associated Hilbert space for this kernel is $\mathbb{R}^{|\Sigma^p|}$ - the space of all possible
strings over $\Sigma$ of length $p$. The p-spectrum string kernel can be efficiently implemented by using \textit{trie} data-structures and \textit{n-gram} models.

\subsubsection{Co-reference Resolution by String Kernels}
\label{sec:co_reference_resolution_by_string_kernels}
Co-reference resolution can be divided into three consecutive subtasks \cite{string_kernel_coref}:
\begin{itemize}
  \item Binding constrains
  \item Expletive detection
  \item Aliasing
\end{itemize}
For all of these subtasks tailored kernel methods have been proposed.

Co-refernce resolution by kernel methods is an active research area and several methods have been proposed for the task. These include string and tree based kernel methods \cite{string_kernel_coref}.
The advantage of tree based methods is due to the possibility to encode additional structure and information about syntactic and semantic properties of the text, which renders these methods superior to string based methods.

For our project, we need co-reference resolution based on the extracted named entities. That is, we want to resolve references between different named entities. This essentially comes down to the task of \textit{aliasing} named entities for each type of named entity. To this end, we applied a two-spectrum string kernel, where the basic idea is to compute similarities between all named entities and based on that to group similar ones. Grouping is not done by hard clustering, i.e.\ we do not cluster the named entities in different groups but rather use a soft clustering based on the resulting kernel matrix.

\begin{figure}[ht]
  \caption{Example for aliasing of document- or query-vectors by string kernel similarity.
The matrix $\mathcal{S}$ is the \textit{kernel matrix} of the three terms $\mathcal{T} = \lbrace
\text{Fidel Castro},\text{Dr. Fidel Castro}, \text{Raul}\rbrace$. The first and second term have a
very high lexical similarity of $0.8$ while the third term is highly dissimilar. Using
$\mathcal{S}$, a document vector or query $q=(1,0,0)^T$ can now be expanded into an aliased form
$\bar q$, where $\bar q = \mathcal{S}q$. In general, aliasing is performed on all document vectors
and queries.}
  \[
     \underbrace{\begin{pmatrix}
      1   & 0.8 & 0\\
      0.8 & 1   & 0\\
      0   & 0   & 1\\
     \end{pmatrix}}_{\mathcal{S}}
     \underbrace{\begin{pmatrix}
     1\\ 0\\0 
     \end{pmatrix}}_{q}
     =
     \underbrace{\begin{pmatrix}
     1\\ 0.8\\0 
     \end{pmatrix}}_{\bar q}
  \]
  \label{eq:example_string_sim}
\end{figure}

The first implementation of the two-spectrum kernel function was done in \texttt{python} but we  later reimplemented it in \texttt{C++} for performance reasons. For all three types of named entities we computed the complete kernel matrix $\textbf{K}$, where $\textbf{K}_{i,j}$ is the similarity of the named entities $i$ and $j$. Since many of the entries in $\textbf{K}$ are near zero, thresholding with a fixed value was applied after the computation and the resulting sparse matricies where then stored.

Given the kernel matrix $\textbf{K}$, aliasing can be performed through multiplication of each document-vector with the kernel matrix. The scalar product on the resulting aliased document vectors yields the overall document similarity with respect to the named entities. In figure \ref{eq:example_string_sim} we give a short example on aliasing based on three different named entities.