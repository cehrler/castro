In general, the idea of kernel methods is to map objects of a
distinct type $K$ into some \textit{Hilbert space} $\mathcal{H}$ using a mapping function $\Phi :
K \rightarrow \mathcal{H}$. Here, $K$ can be an arbitrary structure like e.g. trees, strings,
graphs, etc. Given such a mapping $\Phi$ the similarity of two objects $s,t \in K$ can be computed
by using the scalar product defined on the Hilbert space:
\[sim(s,t) = <\Phi(s),\Phi(t)>\]
Instead of defining the mapping function $\Phi$ explicitly, a kernel function $k : K \times K
\rightarrow \mathbb{R}$ can be used. Kernel functions are positive-semidefinite functions
who implicitely compute the scalar product of two objects on some Hilbert space. That is, for every
kernel function $k$ there exists a mapping $\Phi$, such that
\[k(s,t) = <\Phi(s),\Phi(t)>.\]
This property of kernel functions is often exploited in practise because it greatly facilitates the
computational cost that is needed to compute object similarity.

\textit{String kernels} are kernel methods \note{cite} that operate on the domain of strings over
an arbitrary finite alphabet $\Sigma$ and are often applied in the fields of \textit{information
retrieval} and \textit{bioinformatics}. For the purpose of our work, we applied a
\textit{p-spectrum} string kernel \note{citattion needed} which is defined by
\[k_p(s,t) = \sum_{u \in \Sigma^p}{\phi_u^p(s)\phi_u^p(t)},\]
where $\phi_u^p(s)$ counts the number of occurences of the substring $u$ of length $p$ in $s$. The
associated Hilbert space for this kernel is $\mathbb{R}^{|\Sigma^p|}$ - the space of all possible
strings over $\Sigme$ of length $p$.

\note{how to compute this efficient}

\note{hot to do coref}

\begin{figure}[ht]
  \caption{Example for aliasing of document- or query-vectors by string kernel similarity.
The matrix $\mathcal{S}$ is the \textit{kernel matrix} of the three terms $\mathcal{T} = \lbrace
\text{Fidel Castro},\text{Dr. Fidel Castro}, \text{Raul}\rbrace$. The first and second term have a
very high lexical similarity of $0.8$ while the third term is highly disimilar. Using
$\mathcal{S}$, a document vector or query $q=(1,0,0)$ can now be expanded into an aliased form
$\bar q$, where $\bar q = \mathcal{S}q$. In general, aliasing is performed on all document vectors
and queries.}
  \[
     \underbrace{\begin{pmatrix}
      1   & 0.8 & 0\\
      0.8 & 1   & 0\\
      0   & 0   & 1\\
     \end{pmatrix}}_{\mathcal{S}}
     \underbrace{\begin{pmatrix}
     1\\ 0\\0 
     \end{pmatrix}}_{q}
     =
     \underbrace{\begin{pmatrix}
     1\\ 0.8\\0 
     \end{pmatrix}}_{\bar q}
  \]
  \label{eq:example_string_sim}
\end{figure}

\note{how we did coref}
\note{cpp implemnetation, simmatrix as sparse matrix... approximation...}

HOW IT RELATES TO FULL COREFF SESTEM, HOW IT RELATES TO OTHER STRING DISTANCE METHODS

IMPLEMENTATION ASPECTS (which package, additional scripting etc.. )

PROBLEMS, SOLUTIONS



