\subsection {Summary}
\label{sec:summary}

We presented a system that bla bla bla
TO BE COMPLETED


\subsection {Future Work}
\label{sec:future_work}
The presented system is a work in constant progress. 
In the following we present several ideas for possible improvements we plan to experiment with in future work.

TO BE EXPANDED
APLICABILITY OF APPROACH TO OTHER HISTORICAL DATABASES

\subsubsection{NLP processing}
For our current goals we limited ourselves to three kinds of named entities. 
It is however possible to extract additional relevant types of named entities. 
This direction can be further expanded with automatic extraction of events and other kinds of information that can be relevant for measuring 
document similarity in a historically motivated way.  

The proposed coreference resolution approach can be further expanded to include referring expressions. 
This is expected to have a positive impact on the reliability of the similarity measure. 
We also intend to experiment with an off-the-shelf coreference resolution system such as BART.
We plan to perform more thorough evaluation of each of the NLP dub tasks. 


\subsubsection{Visualization}
The visualization system can be further improved in several manners. 
A direction we intend to explore is the relevance and possibilities of plotting and representing correlations of the retrieved documents 
to different parameters, such as time and locations.

We also indent to experiment with additional graph layouts that have the potential of highlighting different properties of the 
retrieved set of documents. For example, we plan to implement a chronological layout, in which nodes are presented on a single chronological 
line. Edges will come in the form of arcs. Such representation has the potential of revealing time related information that is not 
directly visible in the current layout. 

As our representation comes in the form of a graph, a natural direction for futher exploration would be to examine the usefulness of 
graph algorithms for inferring interpretable information. One of the possibilities is to experiment with automatic clustering algorithm 
that would enable highligting clusters in the graph. 

Finally, we plan to perform several optimizations that would improve the real-time performance of our graphical user interface.  

We believe that enhancing the system in the proposed directions can further establish the relevance and usefulness of our tool for historians.  

