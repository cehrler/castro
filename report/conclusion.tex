\subsection {Summary}
\label{sec:summary}

We presented a system that supports historians in their work with collections of historical documents. 
Our application provides an innovative interface that enables searching and browsing collections in an intuitive and easy to use way.
Most importantly, our approach aims at helping historians to formalize research hypotheses and testing them on the collection.

We achieve that by using NLP and IR techniques to extract historically relevant information about the documents and by exploiting
this information for determining potential links between documents. We organize the data and present it graphically,
in a way that can reveal hidden patterns and connections between the documents.
 
It is important to stress that though our current version of the system is dedicated solely to the Castro Speech Database, 
it can be relatively easily extended to other collections of historical data. Moreover, Domain Knowledge Visualization literature
provides us with promising indications that our core idea of extracting domain relevant information and organizing documents by similarity 
according to this information can be tailored to collections of texts in other domains. Having said that, our current focus still lies in the realm of historical texts. 

One of the main issues that became clear during this work is that the complexity of our task makes it extremely challenging to evaluate the performance of the system as a whole. The multitude of parameters and the subjective and complex nature of the notions document similarity and document relatedness, especially in the historical domain, make it difficult to determine the validity of our approach in a scientific manner. 

\subsection {Future Work}
\label{sec:future_work}
As our system has many sub-components, improvements and extensions are possible
at almost in all aspects of the project. In the following we present several ideas for possible improvements we plan to experiment with in the
near future.

\subsubsection{NLP and IR Processing}
\label{sec:future_work_NLP}
For our current goals we limited ourselves to three kinds of NEs. 
However, it is possible to extract additional relevant types of NEs. 
Similarly, other types of information could be extracted, such as of events. This additional information might also be interesting for measuring 
document similarity in a historically motivated way.  

As described previously, the string kernel based aliasing method works on the characters level. 
This leads to over-generation which can be only partially reduced by raising the inter-entity similarity threshold.   
A further improvement of the aliasing precision might come in the form of incorporating knowledge about the structure of the NEs. 
For example, a name could be tagged
with labels such as \meta{First Name}, \meta{Last Name}, \meta{Occupation} and more. This knowledge can then be integrated as tree kernels
that would further constrain the acceptable name variations. The feasibility of this direction is exemplified in \cite{string_kernel_coref}.
We could further rely on their approach to expand our mechanism to include referring expressions by using expletive and binding kernels.
Such improvements are extremely important, as they are expected to have a positive impact on the reliability of the similarity measurements and
the quality of the resulting inter-document linking. 
Reliable aliasing is also important for our query expansion mechanism and hence has impact on the quality of the retrieval.
As an alternative to using simply a string kernel, we also intend to experiment with an off-the-shelf CRR system such as 
BART and compare the results of both approaches. 

In the current implementation, we measure document similarity using the Manhattan distance and cosine similarity and approximate document-query similarity 
using the cosine measure. There are, however, other measures which could be tested for both similarity measurements. 

Our current query expansion mechanism is an artifact of the document boosting with aliases and works only for NEs. This query expansion approach can be further applied to terms that are not NEs. However, we could also implement a more direct approach that focuses on the queries rather then on the documents. A few of the ideas for this direction are fuzzy matching of query terms to document terms, lemmatization, expanding queries according to distributional similarity and other techniques. 



\subsubsection{Visualization and GUI Functionalities}

We are interested in experimenting with additional graph layouts that have the potential of highlighting different properties of the 
retrieved set of documents. For example, we plan to implement a chronological layout, in which nodes are presented on a single chronological 
line. Edges will come in the form of arcs. Such representation has the potential of unveiling time related information that is not 
directly visible in the current layout. Another possibility would be arranging the documents according to their location, possibly locating each on an image of a map. 

Currently, we present a table in which each line contains metadata information about a certain document. This representation can be extended
to include Google-style text snippets of each document, with highlighted query terms.

An important extension that would increase the generality and usefulness of the system, 
is an option for loading and switching between several databases. This will allow the user to work with several collections. Querying options
would then have to be adjusted to the metadata available for each collection.

In addition to the major planned extensions, we are continuing to experiment with different designs and features that are meant to make the
system more intuitive and simple to use. 

\subsubsection{Performance Optimization and Scalability}
Our system is designed to be a practical application which history students and historians can use in their daily study and research activities.
For this goal to be achieved, the system has to be optimal in terms of loading time and GUI performance and be scalable for collections 
of very large size. We have already done several optimizations that improved the system performance compared to
previous versions. Among others, we optimized the graph components by elimination of unnecessary graph redrawing. 
Index files and and similarity matrices files were converted to a binary format which optimizes loading time and space requirements. 
To enhance the performance of the application further and make the system more stable, we intend to make use of multi-threading techniques. 

Our final goal is to achieve fast loading time of the application, fast retrieval times of the documents and completely 
smooth performance of all GUI components over multiple queries. We will test performance on collections larger then the Castro Speech Database to ensure
that the system is indeed scalable.    

\subsubsection{Evaluation}

We have currently performed a relatively small scale and mostly qualitative evaluation of our system. We would like to 
expand and systematize the evaluation process in order to be able to more easily identify sources of problems and further improvements.
In particular, we plan to test in a more controlled manner whether our similarity measurements according to NEs indeed correlate with human judgements of similarity and relatedness.   

One of the central aspects of our future evaluation scheme will be user evaluation and feedback. We intend to introduce our system
to students and researchers in history departments and receive feedback on their experiences with the system.
To this end we will design questionnaires that will guide the users with the testing of the application and help us organize their feedback.
 
We believe that evaluating and enhancing the system in the proposed directions can further establish the potential relevance and usefulness 
of our tool for historians.  

