The query functions, graph and display functions are integrated into a single graphical front-end relying primarily on the mouse for user input.

\begin{figure}[h]
\centering
\caption{GUI}
%\includegraphics[width=160mm]{gui.png}
\end{figure}

\subsubsection{Search function}
\begin{figure}[h]
\centering
\caption{GUI}
%\includegraphics[width=160mm]{search.png}
\end{figure}

Search terms are entered in \note{picture ref}: Entering any term will return both NEs and also keywords sharing the exact form (not case-sensitive)--- For instance, entering \lingform{committee} will both return exact instances of \code{committee} in the document itself, and will return NEs similar to \lingform{committee} based on the string kernel similarity measure, e.g.\ \lingform{Central Committee} and \lingform{Central Committee of the Cuban Communist Party}. Multi-word searches are denoted in double quotes, e.g.\ \code{``Central Committee''}. The maximum amount of query results presented are specified by \note{picture ref}: For example, entering a value of \code{10} will return ten documents most relevant to the query. The set of years searched between is specified by \note{picture ref}. The type(s) of documents searched (e.g.\ \code{Speech}, \code{Interview}) is specified in \note{picture ref}. Once the criteria are entered, the query is run by \note{picture ref}.

\subsubsection{Results}
\begin{figure}[h]
\centering
\caption{GUI}
%\includegraphics[width=160mm]{representation.png}
\end{figure}

\paragraph{Presentation}
The results of the current query are displayed both in a table of results (\note{picture ref}) and in a graph displayed in the main window (\note{picture ref}).

In the table, the documents are displayed in rows sorted by their relevance to the query in descending order. The metadata associated with each document is displayed in columns.

\subsubsection{Results}
\begin{figure}[h]
\centering
\caption{GUI}
%\includegraphics[width=160mm]{nodecloseup.png}
\end{figure}

In the graph, each document is represented by a node \note{picture ref}. The relevance of the document to the query is represented by the size of the node: The larger the node, the more relevant the document is to the submitted query. The edges between the nodes represent similarities between the documents based on a particular NE the documents have in common, or two similar NEs based on the string kernel measure. The stronger the similarity connection, the heavier the line weight of the edge drawn. There are three line weights: ``thick'' \note{picture ref}, ``normal'' \note{picture ref} and ``dotted'' \note{picture ref}, in descending order of the level of similarity they represent.

\paragraph{Navigation}
Selecting either an entry in the table or a node in the graph with the mouse or keyboard displays the NEs in the document in a sidebar. The NEs in a particular document and their string kernel-based aliases are displayed in \note{picture ref}: Boldface entries represent NEs in the given document, while the non-boldface entries represent NEs similar to the boldface NEs in the document according to the string kernel similarity measure. For example, the NE \textbf{\lingform{Castro}} may also return the similar NEs \lingform{Fidel}, \lingform{Fidel Castro} and \lingform{Dr. Fidel}. The type of the NE (e.g.\ \code{Person} or \code{Organization}) is represented by the color of the text, e.g.\ red for \meta{Persons}, green for \meta{Organizations} and blue for \meta{Locations}. It also is possible to select multiple documents, e.g.\ by holding SHIFT and clicking on multiple nodes/table entries, or by clicking and dragging the mouse to create a selection box encompassing the nodes/table entries to be selected. When multiple documents are selected, the entries displayed in \note{picture ref} are those that are common to all selected documents.

\begin{figure}[h]
\centering
\caption{GUI}
%\includegraphics[width=160mm]{nodeselect.png}
\end{figure}

In addition to viewing document metadata in the query results table, the user may to view the metadata of the document associated with a particular node (e.g.\ document \meta{Type}, \meta{Date} or \meta{Location}) through the node's context menu (i.e.\ by right-clicking on the node with the mouse on \note{picture ref}). Also available through the context menu is the text of the document itself (\note{picture ref}), which is displayed in a separate window which can be kept open while still navigating through the graph and viewing other nodes. The NEs in the text are denoted by being colored according to their type. Lastly, it is possible to automatically select the nearest neighbors of the node (\note{picture ref}), that is, to mark the nodes which share a named entity directly with the said node.