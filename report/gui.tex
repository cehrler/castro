\subsection {Graphical Representation}
\label{sec:graphical_representation}
\note general idea, what information if provided

Using the similarity matrix generated, a graph of document similarities is generated: Each document is represented as a node, which is connected to other documents in the graph by edges representing the similarity measure of the two documents. Clicking on the node displays the named entities associated with the document the node represents in a sidebar. A boldface entries represent a named entity in the given document, while the non-boldface entries below it represent named entities similar to the boldface named entity according to the string kernel similarity measure. For example, the named entity \scare{\code{Castro}} may

Such a graph is generated from the results of a database query supplied by the user through a text field in the graphical user interface. The documents most relevant to the current query are displayed, the number of the documents displayed being specified by the user (e.g. display the 25 most relevant documents), and edges connecting each document are drawn based on the similarity measures relating each document to each other.

By default, the edges drawn are determined by the absolute similarity measure: If the similarity measure of two documents is greater than a user-adjustable edge threshold, an edge is drawn between them. The higher the absolute similarity of the two documents, the thicker the edge.

However, the user may set the method of edge generation to display edges according to the relative similarity of the documents shown: According to the user-specifiable edge density, the more similar two documents are compared to the similarities of other documents, the thicker the edge.

By adjusting the edge threshold (for absolute similarity) or the edge density (for relative similarity) the user can adjust the level of detail represented by the graph. This allows the user to both discern relationships between documents which are not very similar to each other, and to prune out relatively weaker relationships between documents which are very similar to each other.

Additionally, a vertex filter may be applied, which only displays the